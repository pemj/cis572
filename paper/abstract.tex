\begin{abstract}
problem, motivation, solution, takeaway

In this paper, we consider the problem of automatic forest coverage 
classification.  In short, we wish to construct a model capable of 
reliably predicting the predominant type of tree cover for a small 
region of forest, given a collection of cartographic variables. We 
create such a model by leveraging the Random Forest machine learning 
technique.  We supplement this technique by carrying out a number of 
feature transformations and re-encodings, taking into account a number 
of statistical relationships discovered through judicious use of the 
graphical data analysis.

After carrying out such modifications to the feature set, a random 
forest model was found to reach PLACEHOLDER\% accuracy levels, a gain of 
PLACEHOLDER\ percentage points of the next most effective model tested 
on our data set.  The structure of a random forest 
model\cite{breiman2001random} is such that we gain some insight into 
what data trends are most closely correlated with certain results.  
From this data, we can draw certain preliminary conclusions regarding 
the relationships between a number of geological features.  While this 
information is not in itself interesting from a computational 
standpoint, it does have a number of exciting applications in a number 
of interdisciplinary research projects within both the biological and 
geological sciences.
\end{abstract}
%%% Local Variables: 
%%% mode: latex
%%% TeX-master: "main"
%%% End: 
