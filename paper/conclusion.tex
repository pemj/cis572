\section{Conclusion}
\label{sec:-conc}

In this paper, we examine a number of machine learning algorithms as 
they apply to the problem of classifying forest cover based on 
cartographic variables.  We specify accuracy both for naive solutions 
(before tuning), and for solutions with more carefully tuned 
hyperparameters.  The final collection of k-Nearest Neighbor, Support 
Vector Machine, Gradient Boosting Machine, and Random Forest result in 
a significant increase in accuracy over the next best approach, at a 
final accuracy of 80\%.

In the course of our experimentation, we discovered a number of 
interesting relationships between the variables in the feature set.  
If we had the opportunity to undertake future work on this project, we 
would like to undertake a more comprehensive and formal analysis of the 
data.  For instance, what kind of relationship exists between the 
vairous Hillshade values?  How closely correlated are they with 
Elevation?  We have a great many soil types, but some of them seem to 
overlap. Do all soil types classified as ``extremely stony'' occur with 
similar environments, regardless of whether the soil is Bross family or 
Leighcan-Moran family?  We would be interested in answering these 
questions and more, perhaps in conjunction with an actual expert in 
the domain of geology.






%%% Local Variables: 
%%% mode: latex
%%% TeX-master: "main"
%%% End: 
