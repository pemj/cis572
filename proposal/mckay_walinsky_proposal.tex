\documentclass{article}
\usepackage{color}
\usepackage{graphicx}
\usepackage{fullpage}
\usepackage{amssymb}
\usepackage{amsmath}
\usepackage{enumerate}
\usepackage{hyperref}
\pagestyle{empty}

\begin{document}

\begin{center}
Peter McKay, Daniel Walinsky\\
2015-02-19\\
\Large Project Proposal
\end{center}

\bigskip
For our final project, we intend to tackle the following Kaggle problem:
the 
\href{http://www.kaggle.com/c/forest-cover-type-prediction}{Forest Cover Type Prediction}.
In brief, we wish to, given a collection of geographic and cartographic 
variables that apply to a given chunk of forest, determine what kind of 
trees will be most common in that forest region.  

Technically speaking, this is a multi-class classification problem.  


Random forests and 
decision trees would be poetically appropriate, but possibly less than 
optimal.

intro

related work

methodology
We plan to carry out most of our research using Python.  More 
specifically, Python3, with the help of a number of libraries designed 
to assist in machine learning and data analysis.  

\textbf{pandas: Python Data Analysis Library:}
We will make use of data structures and input mechanisms introduced in 
pandas.

\textbf{scikit-learn:}
In an effort to avoid duplicating the wheel (and running out of RAM 
while we try to get that wheel to load), we will be making use of 
scikit-learn to actually carry out our machine algorithms.

\textbf{numpy:}
numpy provides a number of useful 



results

reflection

conclusion


\end{document}
