\documentclass{article}
\usepackage{color}
\usepackage{graphicx}
\usepackage{fullpage}
\usepackage{amssymb}
\usepackage{amsmath}
\usepackage{enumerate}
\usepackage{hyperref}
\pagestyle{empty}

\begin{document}

\begin{center}
Peter McKay, Daniel Walinsky\\
2015-02-19\\
\Large Project Proposal
\end{center}

\bigskip
For our final project, we intend to tackle the following Kaggle problem:
the 
\href{http://www.kaggle.com/c/forest-cover-type-prediction}{Forest Cover Type Prediction}.
In brief, we wish to, given a collection of geographic and cartographic 
variables that apply to a given 30 x 30 meter chunk of forest, determine 
what kind of tree coverage will be most prevalent in that region.

Examples of data fields we have access to include the soil type 
(Cryaquolis, Granile, Rock outcrop), the elevation, and average slope.

We are faced, then, with a multi-class classification problem. From such 
data fields in our testing file, we wish to discover whether a 
particular cell can be categorized as
Spruce/Fir,
Lodgepole Pine,
Ponderosa Pine,
Cottonwood/Willow,
Aspen,
Douglas-fir, or
Krummholz.


Our first step must be to acquire the data and examine it, in order to 
discover what sort of model would best suit our data.  To that end, we 
must first settle upon a set of tools to assist us in this matter.  
We plan to carry out most of our research using Python.  More 
specifically, Python 3.4.2, with the help of a number of libraries designed 
to assist in machine learning and data analysis.  

\textbf{pandas: Python Data Analysis Library:}
We will make use of data structures and input mechanisms introduced in 
pandas.  These data structures, while containing a great deal of 
complex functionality, extend iterables with sufficient cleanliness to 
allow for compatibility with most other python libraries.

\textbf{matplotlib:} 
The first of these such libraries that sounds helpful is matplotlib.  
matplotlib extends python with some matlab-like functionality, allowing 
us to generate graphs and examine the data in question.  Should we need 
to scale the data, or remove a number of corrupted entries, matplotlib 
should give us an idea of where to start .

\textbf{numpy:}
numpy provides a collection of very useful tools for analysis, which 
will come in handy as we examine the 




\textbf{scikit-learn:}
In an effort to avoid duplicating the wheel (and running out of RAM 
while we try to get that wheel to load), we will be making use of 
scikit-learn to actually carry out our machine algorithms.

Random forests and 
decision trees would be poetically appropriate, but possibly less than 
optimal.

intro

related work

methodology






results

reflection

conclusion


\end{document}
